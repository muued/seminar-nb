\documentclass{article}

\usepackage{ucs} % Unicode - dependency of utf8x inputenc
\usepackage[utf8x]{inputenc} % "utf8x" uses "ucs"-package, better than "UTF8"
\usepackage[T1]{fontenc}
\usepackage{lmodern}

\usepackage{amsmath}
\usepackage[german]{babel}
\usepackage{amssymb}
\usepackage{amsxtra}
\usepackage{epsfig,psfrag}
\usepackage{listings}

\newcommand{\refchapter}[1]{Kapitel~\ref{#1}}
\newcommand{\refsec}[1]{Sektion~\ref{#1}}
\newcommand{\refeqn}[1]{Gleichung~(\ref{#1})}
\newcommand{\reffig}[1]{Abbildung~\ref{#1}}

\title{
{\bf \scriptsize RHEINISCH-WESTF\"ALISCHE TECHNISCHE HOCHSCHULE AACHEN \\
LuFG Informatik 12 (Prof. Dr. rer. nat. Uwe Naumann)}
\vspace{.5cm} \\
\epsfig{file=figures/STCE_Logo_WWW.eps,width=.7\textwidth}
\vspace{1cm} \\
{\bf \Large Numerische Bibliotheken} \\
{\bf \large Statistik} \\
{\large Seminararbeit} 
}

\author{Christian Janßen (302530) \\ Fabian Ohler (280424) }

\begin{document}

\lstloadlanguages{[ISO]C++}
\lstset{basicstyle=\small, numbers=left, numberstyle=\footnotesize,
  stepnumber=1, numbersep=5pt, breaklines=true, escapeinside={/*@}{@*/}}

\begin{titlepage}
\thispagestyle{empty}
\maketitle
\end{titlepage}

\pagestyle{headings}

\section{Einführung}
kurze Einführung in das Thema Statistik allgemein, Betonung der Wichtigkeit von Statistik
Überleitung zu Anpassungsgüte-Tests als interessanter Teilaspekt der Statistik, Beschreibung
-> Numerische Bibliotheken machen das Leben schöner
exemplarische Darstellung von $\chi^2$-Test und Kolmogorow-Smirnow-Test (KS-Test)

\section{Problembescheibung}
Seminar soll sich mit Einarbeitung in die o.g. Algorithmen befassen
Dabei soll dargelegt werden, wie die Routinen zu den Algorithmen aufgerufen werden können
Die Eigenschaften der zugrundeliegenden Algorithmen sollen \emph{anhand von Beispielen demonstriert werden}
Alternative Bibliotheken (R (open-source), Matlab (kommerziell), Mathematica (kommerziell)) zu NAG C (kommerziell).
Unterschiede Präzision und Laufzeit?

\section{Arbeitsplan}
% 6 Wochen
\subsection{Anfangsphase (2 Wochen)}
Verfassen von Einführung in Statistikgrundlagen
Einarbeiten in NAG C Libraries, R, Mathematica, Matlab
Implementation von Grundgerüst zum Evaluieren der Algorithmen mit NAG C Libraries, R, Matlab, Mathematica

\subsection{Transfer- und Vergleichsphase (3 Wochen)}
Ausarbeitung von Beispielen
Evaluierung der verschiedenen Bibliotheken
signifikante Unterschiede erkennen und herausarbeiten, im Zweifel vertiefte Literaturrecherche zur Feststellung der Gründe

\subsection{Finalisierungsphase (1 Woche)}
Verfassen von Fazit, Ausblick, ...
Abschließende Arbeiten am Dokument

\nocite{}
\bibliographystyle{plain}
\bibliography{report}

\end{document}

