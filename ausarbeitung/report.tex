\documentclass{article}

\usepackage{ucs} % Unicode - dependency of utf8x inputenc
\usepackage[utf8x]{inputenc} % "utf8x" uses "ucs"-package, better than "UTF8"
\usepackage[T1]{fontenc}
\usepackage{lmodern}

\usepackage{amsmath}
\usepackage[german]{babel}
\usepackage{amssymb}
\usepackage{amsxtra}
\usepackage{epsfig,psfrag}
\usepackage{listings}

\usepackage{epstopdf}
\newcommand{\refchapter}[1]{Kapitel~\ref{#1}}
\newcommand{\refsec}[1]{Sektion~\ref{#1}}
\newcommand{\refeqn}[1]{Gleichung~(\ref{#1})}
\newcommand{\reffig}[1]{Abbildung~\ref{#1}}

\title{
{\bf \scriptsize RHEINISCH-WESTF\"ALISCHE TECHNISCHE HOCHSCHULE AACHEN \\
LuFG Informatik 12 (Prof. Dr. rer. nat. Uwe Naumann)}
\vspace{.5cm} \\
\epsfig{file=figures/STCE_Logo_WWW.eps,width=.7\textwidth}
\vspace{1cm} \\
{\bf \Large Numerische Bibliotheken} \\
{\bf \large Statistik} \\
{\large Seminararbeit} 
}

\author{Christian Janßen (302530) \\ Fabian Ohler (280424) }

\begin{document}

\lstloadlanguages{[ISO]C++}
\lstset{basicstyle=\small, numbers=left, numberstyle=\footnotesize,
  stepnumber=1, numbersep=5pt, breaklines=true, escapeinside={/*@}{@*/}}

\begin{titlepage}
\thispagestyle{empty}
\maketitle
\end{titlepage}

\pagestyle{headings}
\newpage
%%%----Einführung----%%%
\section{Einführung}
Die Statistik als Teilgebiet der Mathematik beschreibt das Sammeln, Auswerten und Beurteilen von Informationen und Daten.
Zusätzlich dient sie der Stochastik als Verbindung zwischen empirischen Daten und den dazu erhobenen theoretischen Modellen.\\
Erhobene Daten helfen zukünftige Ereignisse besser zu prognostizieren, wie zum Beispiel das Wetter oder das Kaufverhalten der Bevölkerung während einer wirtschaftlichen Rezession.\\
Auch hilft die Statistik dabei, aus diskreten Datensätzen Modelle zu erstellen, welche die Verteilung der Werte als Funktion erklärt und somit Rückschlüsse auf Bereiche zulässt, für die keine Daten existieren. Dies ist hilfreich um beispielsweise die Fernsehquote der Gesamtbevölkerung auf Basis einiger weniger Fernsehzuschauer zu erstellen.\\
Da sich zu empirischen Datensätzen nicht jedes theoretische Modell eignet, müssen diese auf ihre Korrektheit im Sinne der Genauigkeit geprüft werden. Je besser die Genauigkeit eines statistischen Modells beziehungsweise seine Fähigkeit einen Datensatz zu erklären, desto besser ist seine Anpassungsgüte oder auch Anpassung.\\
Die Prüfung der Anpassung übernehmen sogenannte Anpassungsgüte-Tests. Diese Tests sind essentiell um die Wahrscheinlichkeit falscher Prognosen zu minimieren und die Qualität statistischer Modelle zu erhöhen.\\
Da diese Tests in der Regel alle vorhandenen empirischen Daten zur Berechnung der Anpassungsgüte verwenden, sind diese mitunter ziemlich aufwendig und zeitintensiv. Um die Berechnung der Anpassungsgüte-Tests zu beschleunigen und den um den Prozess zu automatisieren, existieren diverse numerische Bibliotheken für verschiedene Programmiersprachen.\\
Diese Bibliotheken beinhalten eine Vielzahl von Algorithmen, zur Bewältigung mathematischer Probleme. Sie werden genutzt für die Berechnung von Integralen, der Lösung von Differentialgleichungen oder der Berechnung der Anpassungsgüte eines statistischen Modells.\\
Numerische Bibliotheken ersparen dem Programmierer somit das Schreiben eines eigenen Algorithmus' für mathematische Probleme. Jedoch muss beachtet werden, dass diese Algorithmen, aufgrund der Fähigkeit von Computern Zahlen nur endlich darstellen zu können, bei Eingaben einen Fehler aufweisen können, welcher nicht immer unerheblich ist.\\
Exemplarisch sollen in diesem Seminar zwei Anpassungsgüte-Test, namentlich der $\chi^2$-Test und der Kolmogorow-Smirnow-Test, mitsamt ihren Implementierungen in numerischen Bibliotheken erklärt werden.
\\1\\
2\\
3\\
4\\
5\\
6\\
7

%%%--------------------------------------
\newpage
\section{Einführung}
kurze Einführung in das Thema Statistik allgemein, Betonung der Wichtigkeit von Statistik
Überleitung zu Anpassungsgüte-Tests als interessanter Teilaspekt der Statistik, Beschreibung
-> Numerische Bibliotheken machen das Leben schöner
exemplarische Darstellung von $\chi^2$-Test und Kolmogorow-Smirnow-Test (KS-Test)

\section{Problembescheibung}
Seminar soll sich mit Einarbeitung in die o.g. Algorithmen befassen
Dabei soll dargelegt werden, wie die Routinen zu den Algorithmen aufgerufen werden können
Die Eigenschaften der zugrundeliegenden Algorithmen sollen \emph{anhand von Beispielen demonstriert werden}
Alternative Bibliotheken (R (open-source), Matlab (kommerziell), Mathematica (kommerziell)) zu NAG C (kommerziell).
Unterschiede Präzision und Laufzeit?

\section{Arbeitsplan}
Für die Bearbeitung des Seminars müssen die verschiedenen Aufgaben in eine sinnvolle zeitliche Abfolge gebracht und ein Plan über den geschätzten Zeitaufwand erstellt werden.
Die für die Seminararbeit zur Verfügung stehenden Sechs Wochen sollen dabei sinnvoll eingeteilt werden.
Zusätzlich soll darauf eingegangen werden, wie auf das Wegfallen eines Gruppenmitglieds reagiert werden kann.
Das Ergebnis dieser Überlegungen ist ein Arbeitsplan, der den nun im Folgenden näher erläutert werden soll.
% 6 Wochen
\subsection{Anfangsphase (2 Wochen)}
Verfassen von Einführung in Statistikgrundlagen
Einarbeiten in NAG C Libraries, R, Mathematica, Matlab
Implementation von Grundgerüst zum Evaluieren der Algorithmen mit NAG C Libraries, R, Matlab, Mathematica

\subsection{Transfer- und Vergleichsphase (3 Wochen)}
Ausarbeitung von Beispielen
Evaluierung der verschiedenen Bibliotheken
signifikante Unterschiede erkennen und herausarbeiten, im Zweifel vertiefte Literaturrecherche zur Feststellung der Gründe

\subsection{Finalisierungsphase (1 Woche)}
Verfassen von Fazit, Ausblick, ...
Abschließende Arbeiten am Dokument

\nocite{}
\bibliographystyle{plain}
\bibliography{report}

\end{document}

