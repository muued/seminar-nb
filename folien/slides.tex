\documentclass{beamer}
\usetheme{STCE}
\usepackage{beamerfoils}

\usepackage{epstopdf}

\setbeamertemplate{footline}[frame number]

\MyLogo{
\centering
\includegraphics[width=0.12\textwidth]{./figures/logo.eps}
}

\begin{document}
\title{\centering
\includegraphics[width=0.25\textwidth]{./figures/logo.eps} \\ Numerische Bibliotheken}

\subtitle{Der $\chi^2$ und Kolmogorow-Smirnow Test}
\author{Christian Janßen (302530) \\ Fabian Ohler (280424) }
\date{August 6, 2013}
\institute{
LuFG Informatik 12: Software and Tools for Computational Engineering}
\frame[plain]{\titlepage}

%%% Inhalt
\begin{frame}
\frametitle{Inhalt}
	\begin{itemize}
		\item Statistische Grundlagen
%		\begin{itemize}
%			\item test 1.2
%			\item test 1.1
%		\end{itemize}
		\item Kolmogorow-Smirnow Test
		\begin{itemize}
			\item Mathematische Hintergrund
			\item Implementierung in NAG C
			\item Alternative Bibliotheken
		\end{itemize}
		\item $\chi^2$ Test
		\begin{itemize}
			\item Mathematische Hintergrund
			\item Implementierung in NAG C
			\item Alternative Bibliotheken
		\end{itemize}
		\item Live Demo
		\item Zusammenfassung
	\end{itemize}
\end{frame}

%%%Statistische Grundlagen
\begin{frame}
\frametitle{Statistische Grundlagen}
\end{frame}

%%%Kolmogorow-Smirnow Test
\begin{frame}
\frametitle{Kolmogorow-Smirnow Test}
Kolmogorow-Smirnow Test hat zwei Anwendungen:\\
	\begin{itemize}
		\item[1.] Pr\"uft Mengen $X$ und $ Y$ auf gleiche Wahrscheinlichkeitsverteilung \\
		\item[2.] Pr\"uft Menge $X$ auf spezifizierte Wahrscheinlichkeitsverteilung $F(x)$ \\
			\[S(x)=F(x)\]
			\begin{center}$\rightarrow$ Anpassungsg\"utetest\end{center}
	\end{itemize}
\end{frame}
	%%%Mathematischer Hintergrund
\begin{frame}
\frametitle{Kolmogorow-Smirnow Test}
Anwendung des Anpassungsg\"utetests:
	\begin{itemize}
		\item Menge $X$ aufsteigend sortieren, $x_1\le ...\le x_n$
		\item<2-> Bilden der empirischen Verteilungsfunktion $S(x)$	
		\only<2>{ $$S(x)=\frac{1}{n}\sum^{n}_{i=1}I_{x_i\le x}$$}
		\item<3-> Spezifizieren der angenommenen Verteilungsfunktion $F(x)$
		\item<3-> Aufstellen der Nullhypothese $H_0$ und der Alternativhypothese $H_1$
		\only<3>{$$H_0:S(x)=F(x)$$ $$H_1:S(x)\neq F(x)$$}
		\item<4-> Ermitteln der Teststatistik $D_n$
		\only<4>{ $$D_n=\max(d_0,d_1)$$ $$ d_0=\max_x|S(x)-F(x)|,$$ $$d_1=\max_x|S(x_{i-1})-F(x_i)|$$ }
		\item<5->Wenn $D_n>D_\alpha$ wird die Nullhypothese verworfen
	\end{itemize}
\end{frame}
	%%%Beispie
\begin{frame}
\frametitle{Kolmogorow-Smirnow Test}
Ein Beispiel:
\end{frame}
	%%%NAG C
\begin{frame}
\frametitle{Kolmogorow-Smirnow Test}
\end{frame}
	%%%R
\begin{frame}
\frametitle{Kolmogorow-Smirnow Test}
\end{frame}

%%%$\chi^2$ Test
\begin{frame}
\frametitle{$\chi^2$ Test}
\end{frame}

%%%Live Demo
\begin{frame}
\frametitle{Live Demo}
\center
\Huge Live Demo
\end{frame}

%%%Zusammenfassung
\begin{frame}
\frametitle{Zusammenfassung}
\end{frame}


\end{document}
