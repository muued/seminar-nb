\documentclass{beamer}
\usetheme{STCE}
\usepackage{beamerfoils}

\usepackage{epstopdf}

\usepackage{ucs} % Unicode - dependency of utf8x inputenc
\usepackage[utf8x]{inputenc} % "utf8x" uses "ucs"-package, better than "UTF8"
\usepackage[T1]{fontenc}
\usepackage{lmodern}

\usepackage[german]{babel}
\usepackage{amsmath, amssymb, amsxtra, amsthm}
\usepackage{epsfig,psfrag, epstopdf}
\usepackage{listings}

\usepackage{tikz}

\usepackage{nicefrac}


\beamertemplatenavigationsymbolsempty % turn off navigation bar


\setbeamertemplate{footline}[frame number]

\MyLogo{
\centering
\includegraphics[width=0.12\textwidth]{./figures/logo.eps}
}

\begin{document}
\title{\centering
\includegraphics[width=0.25\textwidth]{./figures/logo.eps} \\ Numerische Bibliotheken}

\subtitle{Der $\chi^2$ und Kolmogorow-Smirnow Test}
\author{Christian Janßen (302530) \\ Fabian Ohler (280424) }
\date{6. August 2013}
\institute{
LuFG Informatik 12: Software and Tools for Computational Engineering}
\frame[plain]{\titlepage}

%%% Inhalt
\begin{frame}
\frametitle{Inhalt}
	\begin{itemize}
		\item Statistische Grundlagen
%		\begin{itemize}
%			\item test 1.2
%			\item test 1.1
%		\end{itemize}
		\item Kolmogorow-Smirnow Test
		\begin{itemize}
			\item Mathematische Hintergrund
			\item Implementierung in NAG C
			\item Alternative Bibliotheken
		\end{itemize}
		\item $\chi^2$ Test
		\begin{itemize}
			\item Mathematische Hintergrund
			\item Implementierung in NAG C
			\item Alternative Bibliotheken
		\end{itemize}
		\item Live Demo
		\item Zusammenfassung
	\end{itemize}
\end{frame}

%%%Statistische Grundlagen
\begin{frame}
\frametitle{Statistische Grundlagen}
\end{frame}

%%%Kolmogorow-Smirnow Test
\begin{frame}
\frametitle{Kolmogorow-Smirnow Test}
Kolmogorow-Smirnow Test hat zwei Anwendungen:\\
	\begin{itemize}
		\item[1.] Pr\"uft Mengen $X$ und $ Y$ auf gleiche Wahrscheinlichkeitsverteilung \\
		\item[2.] Pr\"uft Menge $X$ auf spezifizierte Wahrscheinlichkeitsverteilung $F(x)$ \\
			\[S(x)=F(x)\]
			\begin{center}$\rightarrow$ Anpassungsg\"utetest\end{center}
	\end{itemize}
\end{frame}
	%%%Mathematischer Hintergrund
\begin{frame}
\frametitle{Kolmogorow-Smirnow Test}
Anwendung des Anpassungsg\"utetests:
	\begin{itemize}
		\item Menge $X$ aufsteigend sortieren, $x_1\le ...\le x_n$
		\item<2-> Bilden der empirischen Verteilungsfunktion $S(x)$	
		\only<2>{ $$S(x)=\frac{1}{n}\sum^{n}_{i=1}I_{x_i\le x}$$}
		\item<3-> Spezifizieren der angenommenen Verteilungsfunktion $F(x)$
		\item<3-> Aufstellen der Nullhypothese $H_0$ und der Alternativhypothese $H_1$
		\only<3>{$$H_0:S(x)=F(x)$$ $$H_1:S(x)\neq F(x)$$}
		\item<4-> Ermitteln der Teststatistik $D_n$
		\only<4>{ $$D_n=\max(d_0,d_1)$$ $$ d_0=\max_x|S(x)-F(x)|,$$ $$d_1=\max_x|S(x_{i-1})-F(x_i)|$$ }
		\item<5->Wenn $D_n>D_\alpha$ wird die Nullhypothese verworfen
	\end{itemize}
\end{frame}
	%%%Beispie
\begin{frame}
\frametitle{Kolmogorow-Smirnow Test}
Ein Beispiel:\\
	\begin{itemize}
		\item Beobachtungen:
			\begin{table}[h]
			\center
			\begin{tabular}{c|c|c|c|c|c|c|c|c|c}
			$x_1$	&$x_2$	&$x_3$	&$x_4$	&$x_5$	&$x_6$	&$x_7$	&$x_8$	&$x_9$	&$x_{10}$\\
			\hline
			1	&1.1	&1.2	&1.6	&1.7	&2.1	&2.1	&2.4	&2.4	&2.5	\\
			\end{tabular}
			\begin{tabular}{c|c|c|c|c|c|c|c|c|c}
			$x_{11}$	&$x_{12}$	&$x_{13}$	&$x_{14}$	&$x_{15}$	&$x_{16}$	&$x_{17}$	&$x_{18}$	&$x_{19}$	&$x_{20}$\\
			\hline
			2.6	&2.6	&2.6	&2.7	&2.8	&3	&3.3	&3.5	&3.8	&4.2\\
			\end{tabular}
			\end{table}
		\item Hypothesen: $$H_{Franz}:S(x)=\Phi(x|3.5;1)$$ $$H_{Paul}:S(x)=\Phi(x|3;0.7)$$
	\end{itemize}
\end{frame}

\begin{frame}
\frametitle{Kolmogorow-Smirnow Test}
Berechnung der Werte:\\
	\begin{itemize}
		\item Werte der Verteilungsfunktionen
			\begin{table}[ht]
			\center
			\begin{tabular}{c|c|c|c|c}
			$i$ 	& $x_i$ 	& $S(x_i)$ 	& $\Phi (x_i|3;0,7)$ 	& $\Phi (x_i|3,5;1)$ 	\\
			\hline
			1	&	1	&	0.05	&	0.002137	&	0.006210	\\
			12	&	2.6	&	0.65	&	0.283855	&	0.184060	\\
			15	&	2.8	&	0.75	&	0.387548	&	0.241964	\\
			20	&	4.2	&	1	&	0.956762	&	0.758036	\\
			\end{tabular}
			\end{table}
		\item Werte der Teststatistik
			\begin{table}[ht]
			\center
			\begin{tabular}{c|c|c|c|c}
			$i$ 	& $d_{0,Paul}(x_i)$ 	& $d_{1,Paul}(x_i)$ 	& $d_{0,Franz}(x_i)$ 	& $d_{1,Franz}(x_i)$ 	\\
			\hline
			1	&	0.047863	&	0.002137	&	0.043790	&	0.006210	\\
			12	&	0.366145	&	0.366145	&	0.465940	&	0.465940	\\
			15	&	0.362452	&	0.312452	&	0.508036	&	0.458036	\\
			20	&	0.043238	&	0.006762	&	0.241964	&	0.191964	\\
			\end{tabular}
			\end{table}
	\end{itemize}
\end{frame}

\begin{frame}
\frametitle{Kolmogorow-Smirnow Test}
Auswertung der Teststatistiken:\\
	\begin{itemize}
		\item $D_{n,Franz}=0.508036$ und $D_{n,Paul}=0.366145$
		\item $D_\alpha=0.294$
	\end{itemize}
Somit sind beide Hypothesen verworfen.
\end{frame}
	%%%NAG C
\begin{frame}
\frametitle{Kolmogorow-Smirnow Test}
\end{frame}
	%%%R
\begin{frame}
\frametitle{Kolmogorow-Smirnow Test}
\end{frame}

%%%$\chi^2$ Test
\begin{frame}
\frametitle{$\chi^2$ Test}
\end{frame}

%%%Live Demo
\begin{frame}
\frametitle{Live Demo}
\center
\Huge Live Demo
\end{frame}

%%%Zusammenfassung
\begin{frame}
\frametitle{Zusammenfassung}
\end{frame}


\end{document}
